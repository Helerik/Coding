\documentclass[11pt]{article}
\usepackage[margin=1in]{geometry}
\usepackage{amsfonts, amsmath, amssymb}
\usepackage[none]{hyphenat}
\usepackage{fancyhdr}
\usepackage{graphicx}
\usepackage{float}
\usepackage[nottoc, notlot, notlof]{tocbibind}

\pagestyle{fancy}
\fancyhead{}
\fancyfoot{}
\fancyhead[L]{\slshape \MakeUppercase{MAP 2321 - Lista 1}}
\fancyhead[R]{\slshape Name}
\fancyfoot[C]{\thepage}

\setlength{\parindent}{4em}
\setlength{\parskip}{1em}
\renewcommand{\baselinestretch}{1}

\begin{document}

\begin{titlepage}
\begin{center}
\vspace{1cm}
\Large{\textbf{IME - USP}}\\
\vfill
\line(1,0){400}\\[1mm]
\huge{\textbf{MAP 2321 - Técnicas em Teoria de Controle}}\\[3mm]
\Large{\textbf{Lista 1}}\\[1mm]
\line(1,0){400}\\
\vfill
By:\\
Erik Davino Vincent\\
Henrique Reis Berquo\\
Guilherme Byon Cheol Kang\\
\today\\

\end{center}
\end{titlepage}

\setcounter{page}{1}

\section*{Exercício 1}

Uma \textbf{partícula} de massa unitária está sob ação de um campo de aceleração \textbf{central} newtoniano. Além disso temos dois controles independentes, um na direção \textbf{radial} e outro na direção \textbf{tangencial} $u_re_r$ e $u_{\theta}e_{\theta}$ respectivamente.\\

\begin{small}
$e_r,\ e_{\theta}\subset\mathbb{R}^2$ formam um referencial móvel unitário.
\end{small}\\

a) Use a Lei de Newton para justificar a equação abaixo como um modelo para a dinâmica do problema acima descrito:

$$\ddot{r} = (-\frac{k}{r^2} + u_r)e_r + u_{\theta}e_{\theta}$$

b) Use coordenadas polares para re-escrever o modelo na maneira a seguir:

$$\ddot{r} = r\dot{\theta}^2-\frac{k}{r^2} + u_r$$
$$r\ddot{\theta} = -2\dot{r}\dot{\theta} + u_{\theta}$$

c) Suponha $u_r = u_{\theta} = 0$. Determine os valores de $k$ para os quais as órbitas circulares $r(t) = \sigma$ e $\theta(t) = \omega t$ sejam soluções do sistema acima.\\

d) Defina as seguintes variáveis de estado:\\

$x_1 = r - \sigma$, $x_2 = \dot{r}$, $x_3 = \sigma(\theta - \omega t)$ e $x_4 = \sigma(\dot{\theta} - \omega)$

Verifique que a equação linearizada do sistema anterior, tomando $\sigma = 1$, sobre as órbitas circulares é:

$$
\begin{pmatrix}
\dot{x_1}\\ 
\dot{x_2}\\ 
\dot{x_3}\\ 
\dot{x_4}
\end{pmatrix} = \begin{pmatrix}
0 & 1 & 0 & 0\\ 
3\omega^2 & 0 & 0 & 2\omega\\ 
0 & 0 & 0 & 1\\ 
0 & -2\omega & 0 & 0
\end{pmatrix}
\begin{pmatrix}
x_1\\ 
x_2\\ 
x_3\\ 
x_4
\end{pmatrix} + 
\begin{pmatrix}
0 & 0\\ 
1 & 0\\ 
0 & 0\\ 
0 & 1
\end{pmatrix}
\begin{pmatrix}
u_1\\ 
u_2
\end{pmatrix}
$$

e) Calcule:

$$e^{\begin{pmatrix}
0 & 1 & 0 & 0\\ 
3\omega^2 & 0 & 0 & 2\omega\\ 
0 & 0 & 0 & 1\\ 
0 & -2\omega & 0 & 0
\end{pmatrix}(t-t_0)}$$

\section*{Exercício 2}

Sejam $k$ e $b > 0$. Considere o oscilador harmônico dado pela seguinte equação:

$$\ddot{x}+b\dot{x}+kx = \cos t$$

a) Escreva a equação de estado deste sistema e encontre sua solução para qualquer condição inicial dada.\\

b) Seja $\Phi(t,\ t_0)$ a matriz de transição deste sistema. Mostre que:

$$\lim_{t\to +\infty} \Phi(t,\ t_0) = 0$$

c) Prove que existe uma única solução $2\pi$-periódica $\varphi_{2\pi}(t)$ para este sistema.\\

d) Mostre que a solução $2\pi$-periódica do item c) é atratora, i.e., se $x(t)$ é solução,
então $\Vert x(t) - \varphi_{2\pi}(t)\Vert \to 0$ quando $t\to +\infty$.

\section*{Exercício 3}

Seja $x(t)$ a posição de um corpo num instante $t$ sujeito a uma força $f$ dada por:

$$f(t) = \begin{cases} f_0,\ \ t\in[0,t_1]\\
0,\ \ t > t_1 \end{cases}$$

para alguma constante $f_0 > 0$. Se o corpo possui massa $m$ e sofre resistência, temos;

$$m\ddot{x}(t) + b\dot{x}(t) = f(t)$$

para algum $b>0$.\\

a) Calcule a matriz de transição deste sistema.\\

b) Encontre a saída $x(t)$ com condições iniciais $x(0) = 0$ e $\dot{x}(0) = 1$.\\

c) Analise $\lim_{t\to +\infty} x(t)$ assumindo $x(t_0) = x_0$ e  $\dot{x}(t_0) = \dot{x}_0$.

\end{document}











